

%*******************************************************
% This program is free software: you can redistribute it and/or modify
% it under the terms of the GNU General Public License as published by
% the Free Software Foundation, either version 3 of the License, or
% (at your option) any later version.
%
% This program is distributed in the hope that it will be useful,
% but WITHOUT ANY WARRANTY; without even the implied warranty of
% MERCHANTABILITY or FITNESS FOR A PARTICULAR PURPOSE.  See the
% GNU General Public License for more details.
%
% You should have received a copy of the GNU General Public License
% along with this program.  If not, see <http://www.gnu.org/licenses/>.
%*******************************************************
% PhD Thesis Template
% Use XeLatex to typeset
% This template has been tested on
% MacTeX-2015 Distribution
% OS X EI Capitan version 10.11.4
%*******************************************************
% The font is 11pt. Paper size is A4. To be printed
% twoside
%*******************************************************

\documentclass[twoside,paper=a4,fontsize=11pt]{report}

%*******************************************************
% From page 7 of Preparing and Submitting Your
% Thesis - A guide for MPhil and PhD students:
% The thesis submitted for examination shall be typewritten
% or printed on one side or both sides of International size
% A4 paper (except for drawings, maps or tables on
% which no restriction is placed), with a margin of
% not less than 35mm on both right and left-hand
% edges of each page.
% There is no stipulation for the top and bottom margins,
% but it is recommended that they should be 25mm.
% From page10:
% Font heights are usually measured in points and
% the most easily readable fonts are 10 point and 12 point.
%*******************************************************

\usepackage[left=37mm,right=37mm,top=27mm,bottom=27mm]{geometry}

%*******************************************************
% The following three lines set Chinese fonts so that you can
% type your Chinese name.
% The typeset method (so called compiling) is XeLatex.
% Use the Chinese font available on your own computer.
% Yuanti TL Light may not be available on your own computer.
% The scale=1 sets the height of the Chinese font.
%*******************************************************

\usepackage{xltxtra,fontspec,xunicode}
% \usepackage{ctex}
\usepackage{xeCJK}
% \setCJKmainfont[Scale=0.9]{SimSun}
\setCJKmainfont[Scale=0.9]{Songti TC}


%*******************************************************
% The following two lines set the English fonts
%*******************************************************

\setromanfont{Times New Roman}
\setsansfont{Arial}

%*******************************************************
% Include packages that you wish to use.
%*******************************************************

\usepackage{tikz}
\usetikzlibrary{shapes,arrows}

\usepackage{lipsum} % this generates dummy text

\usepackage{multicol} % use multicolumn
\usepackage{rotating} % to create a figure in landscape mode

\usepackage{natbib} % this is for bibliography
\usepackage{setspace} % this is for setting line spacing

\usepackage{longtable} % this is for longtable


\usepackage{pdflscape} % this adds PDF support to the landscape environment of package lscape,

\usepackage{array}


\usepackage{amsthm}
\usepackage{amsmath}
\usepackage{calc}
\usepackage{float}
\usepackage{graphicx}
\usepackage{setspace}
\usepackage{url}

%*******************************************************
% Define hyphenation for some special words
%*******************************************************

\usepackage{hyphenat} % this define hyphenation, an example is given in the next four lines.
\hyphenation{pro-ban-da}
\hyphenation{pro-ban-dum}
\hyphenation{pen-ul-ti-mate}
\hyphenation{sche-ma-tic}


%*******************************************************
% Make "clickable" Table of Contents
%*******************************************************
\usepackage{color}   %May be necessary if you want to color links
\usepackage{hyperref}
\hypersetup{
    %colorlinks=true, %set true if you want colored links
    colorlinks=false
    linktoc=all,         %set to all if you want both sections and subsections linked
    linkcolor=black,  %choose some color, e.g. blue, if you want links to stand out
    }
    
%*******************************************************
% Make glossary
%*******************************************************

\usepackage[acronym,toc]{glossaries}

\makeglossaries
\setacronymstyle{long-short}

\newacronym{ieee}{IEEE}{Institute of Electrical and Electronics Engineers}

\newacronym{ifip}{IFIP}{International Federation for Information Processing}



%*******************************************************
% This is the start of the thesis
%*******************************************************

\begin{document}
	
\linespread{1.2} % change it to 1 becomes single line spacing. Change it to 1.5 is a half line spacing.

%*******************************************************
% From page 10 of Preparing and Submitting Your Thesis
% A guide for MPhil and PhD students:
% Modern word processing programs will automatically
% adjust the line spacing to match the size of text on any line.
% Single line spacing will produce a dense but readable
% text: the text may seem less formidable if one and a
% spacing is used. Double spacing results in a rather
% empty looking page and significantly increases
% the number of pages in a thesis.
%*******************************************************

%*******************************************************
% These are front matters.
% Abstract
% Title page
% Dedication
% Declaration
% Acknowledgment
% Publication (Optional)
% Table of Contents (Including List of Tables and List of Figures)
%*******************************************************


\input{frontMatters/abstract}

\cleardoublepage

\pagenumbering{gobble} % switch of page numbering

\input{frontMatters/titlePage}

\cleardoublepage

\pagenumbering{roman} % switch on page numbering (roman)

\input{frontMatters/dedication}

\cleardoublepage


\pagestyle{plain}

\input{frontMatters/declaration}

\cleardoublepage


%*******************************************************
% This program is free software: you can redistribute it and/or modify
% it under the terms of the GNU General Public License as published by
% the Free Software Foundation, either version 3 of the License, or
% (at your option) any later version.
%
% This program is distributed in the hope that it will be useful,
% but WITHOUT ANY WARRANTY; without even the implied warranty of
% MERCHANTABILITY or FITNESS FOR A PARTICULAR PURPOSE.  See the
% GNU General Public License for more details.
%
% You should have received a copy of the GNU General Public License
% along with this program.  If not, see <http://www.gnu.org/licenses/>.
%*******************************************************
% Acknowledgments
%*******************************************************

\addcontentsline{toc}{chapter}{Acknowledgments}%% to be removed

\chapter*{Acknowledgments}

\begin{flushright}
{\slshape Lorem ipsum dolor sit amet, consectetuer adipiscing.}\\
{\slshape Aenean commodo ligula eget dolor. Aenean massa.}\\% This is a set of dummy text.
\medskip
--- John Doe (1914).% This is a dummy author only.
\end{flushright}

\bigskip

\lipsum[2-5]% This command generate dummy text. Remove this line and replace it by your words of appreciation.

I declare that no part of this thesis has been generated by AI software. These are my own words.

I acknowledge the use of AI technologies in the course of writing this thesis. Table \ref{tab:long} in Appendix \ref{ch:appendixThree} explained the relationship between the use and my work. The AI systems and links were particularised in column 1 of the schedule. The specific uses of Generative AI were particularised in column 2. The prompts used were listed in column 3. I entered those prompts on the dates particularised in column 4. The output from those prompts were particularised in column 5. The outputs from those prompts were used to explain subject matters particularised in column 6. The corresponding section numbers in my work in relation to those subject matters were particularised in column 7.






\cleardoublepage


%*******************************************************
% This program is free software: you can redistribute it and/or modify
% it under the terms of the GNU General Public License as published by
% the Free Software Foundation, either version 3 of the License, or
% (at your option) any later version.
%
% This program is distributed in the hope that it will be useful,
% but WITHOUT ANY WARRANTY; without even the implied warranty of
% MERCHANTABILITY or FITNESS FOR A PARTICULAR PURPOSE.  See the
% GNU General Public License for more details.
%
% You should have received a copy of the GNU General Public License
% along with this program.  If not, see <http://www.gnu.org/licenses/>.
%*******************************************************
% Publication
% Generally, this is necessary if your thesis is organised as a
% series of papers.
% Some say this is optional even though their theses are
% organised as a series of papers.
%*******************************************************
\addcontentsline{toc}{chapter}{Publications}%% to be removed

\chapter*{Publications}

Partial results in this thesis have been published in:

\begin{enumerate}
\item Yufen Chun and John Doe (2015). The Title of the First Work. {\slshape The Name of the First Journal}, 5(3):123 -- 143. (The aforesaid paper was revised and updated as section 2.2 in this thesis)
\item Yufen Chun, Joe Bloggs and John Doe (2016). The Title of the Second work.   {\slshape The Name of the Second Journal}, 6(8):1123 -- 1143. (The aforesaid paper was revised and updated as sections 3.2 and 4.2 in this thesis)
\end{enumerate}


\cleardoublepage
\input{frontMatters/contents}

%*******************************************************
% Mainmatter
%*******************************************************

\pagenumbering{arabic}

\cleardoublepage

\input{chapters/chapter1}

\cleardoublepage

\input{chapters/chapter2}

\cleardoublepage

\input{chapters/chapter3}

\cleardoublepage

\input{chapters/chapter4}

\cleardoublepage

\input{chapters/chapter5}

\cleardoublepage

\input{chapters/chapter6}

\cleardoublepage

\input{chapters/chapter7}

\cleardoublepage

\input{chapters/chapter8}

%*******************************************************
% Backmatter
%*******************************************************

\appendix

\input{appendix/appendChapter1}

\cleardoublepage

\input{appendix/appendChapter2}

\cleardoublepage



%*******************************************************
% This program is free software: you can redistribute it and/or modify
% it under the terms of the GNU General Public License as published by
% the Free Software Foundation, either version 3 of the License, or
% (at your option) any later version.
%
% This program is distributed in the hope that it will be useful,
% but WITHOUT ANY WARRANTY; without even the implied warranty of
% MERCHANTABILITY or FITNESS FOR A PARTICULAR PURPOSE.  See the
% GNU General Public License for more details.
%
% You should have received a copy of the GNU General Public License
% along with this program.  If not, see <http://www.gnu.org/licenses/>.

% \begin{landscape}

%************************************************
\chapter{Generative AI materials}
\label{ch:appendixThree}
%************************************************



\begin{flushright}
{\slshape Lorem ipsum dolor sit amet, consectetuer adipiscing.}\\
{\slshape Aenean commodo ligula eget dolor. Aenean massa.}\\
% This generate dummy text. Remove this line and replace by your quote.
\medskip
--- John Doe,\\
Unified Theory of the Important Theories,\\
{\slshape An Important Journal},\\
Vol.~123, pp.~1234--1243, Dec.~2016.\\
\end{flushright}

\bigskip


\begin{center}
	\begin{longtable}{|l|l|l|l|l|l|l|}
		\caption{Generative AI materials.} \label{tab:long} \\
		
		\hline
		\hline
		\multicolumn{1}{|c|}{\textbf{System \& links}} & \multicolumn{1}{c|}{\textbf{Purposes}} & \multicolumn{1}{c|}{\textbf{Prompts}} &
		\multicolumn{1}{c|}{\textbf{Dates}} &
		\multicolumn{1}{c|}{\textbf{Outputs}} &
		\multicolumn{1}{c|}{\textbf{Subject matters}} &
		\multicolumn{1}{c|}{\textbf{Section numbers}}\\
		
		\hline
		\hline
		\endfirsthead
		
		\multicolumn{7}{c}%
		{\tablename\ \thetable{} -- continued from previous page} \\
		
		\hline
		\hline
		\multicolumn{1}{|c|}{\textbf{System \& links}} & \multicolumn{1}{c|}{\textbf{Purposes}} & \multicolumn{1}{c|}{\textbf{Prompts}} &
		\multicolumn{1}{c|}{\textbf{Dates}} &
		\multicolumn{1}{c|}{\textbf{Outputs}} &
		\multicolumn{1}{c|}{\textbf{Subject matters}} &
		\multicolumn{1}{c|}{\textbf{Section numbers}}\\ \hline
		\endhead
		
		\hline \multicolumn{7}{|r|}{{Continued on next page}} \\ \hline
		\hline
		\endfoot
		
		\hline \hline
		\endlastfoot
		
		One & abcdef ghjijklmn & opqrst & uvwxzy & fedcba & nmlkjijhg & tsrqpo \\
		One & abcdef ghjijklmn & opqrst & uvwxzy & fedcba & nmlkjijhg & tsrqpo \\
		One & abcdef ghjijklmn & opqrst & uvwxzy & fedcba & nmlkjijhg & tsrqpo \\
		One & abcdef ghjijklmn & opqrst & uvwxzy & fedcba & nmlkjijhg & tsrqpo \\
		One & abcdef ghjijklmn & opqrst & uvwxzy & fedcba & nmlkjijhg & tsrqpo \\
		One & abcdef ghjijklmn & opqrst & uvwxzy & fedcba & nmlkjijhg & tsrqpo \\
		One & abcdef ghjijklmn & opqrst & uvwxzy & fedcba & nmlkjijhg & tsrqpo \\
		One & abcdef ghjijklmn & opqrst & uvwxzy & fedcba & nmlkjijhg & tsrqpo \\
		One & abcdef ghjijklmn & opqrst & uvwxzy & fedcba & nmlkjijhg & tsrqpo \\
		One & abcdef ghjijklmn & opqrst & uvwxzy & fedcba & nmlkjijhg & tsrqpo \\
		One & abcdef ghjijklmn & opqrst & uvwxzy & fedcba & nmlkjijhg & tsrqpo \\
		One & abcdef ghjijklmn & opqrst & uvwxzy & fedcba & nmlkjijhg & tsrqpo \\
		One & abcdef ghjijklmn & opqrst & uvwxzy & fedcba & nmlkjijhg & tsrqpo \\
		One & abcdef ghjijklmn & opqrst & uvwxzy & fedcba & nmlkjijhg & tsrqpo \\
		One & abcdef ghjijklmn & opqrst & uvwxzy & fedcba & nmlkjijhg & tsrqpo \\
		One & abcdef ghjijklmn & opqrst & uvwxzy & fedcba & nmlkjijhg & tsrqpo \\
		One & abcdef ghjijklmn & opqrst & uvwxzy & fedcba & nmlkjijhg & tsrqpo \\
		One & abcdef ghjijklmn & opqrst & uvwxzy & fedcba & nmlkjijhg & tsrqpo \\
		One & abcdef ghjijklmn & opqrst & uvwxzy & fedcba & nmlkjijhg & tsrqpo \\
		One & abcdef ghjijklmn & opqrst & uvwxzy & fedcba & nmlkjijhg & tsrqpo \\
		One & abcdef ghjijklmn & opqrst & uvwxzy & fedcba & nmlkjijhg & tsrqpo \\
		One & abcdef ghjijklmn & opqrst & uvwxzy & fedcba & nmlkjijhg & tsrqpo \\
		One & abcdef ghjijklmn & opqrst & uvwxzy & fedcba & nmlkjijhg & tsrqpo \\
		One & abcdef ghjijklmn & opqrst & uvwxzy & fedcba & nmlkjijhg & tsrqpo \\
		One & abcdef ghjijklmn & opqrst & uvwxzy & fedcba & nmlkjijhg & tsrqpo \\
		One & abcdef ghjijklmn & opqrst & uvwxzy & fedcba & nmlkjijhg & tsrqpo \\
		One & abcdef ghjijklmn & opqrst & uvwxzy & fedcba & nmlkjijhg & tsrqpo \\
	
	\end{longtable}
\end{center}

% \end{landscape}


%*****************************************
%*****************************************
%*****************************************
%*****************************************
%*****************************************






\cleardoublepage

%*******************************************************
% Other Stuff in the Back
%*******************************************************
\cleardoublepage

%\bibliographystyle{plainnat}
%\bibliographystyle{plain}

\bibliographystyle{apalike}
\label{app:bibliography} 

\addcontentsline{toc}{chapter}{Bibliography}%% to be removed

\bibliography{exampleBibliography}



%\cleardoublepage\include{FrontBackmatter/Colophon}

%*******************************************************

\end{document}

%*******************************************************
% This is the end of the thesis
%*******************************************************









